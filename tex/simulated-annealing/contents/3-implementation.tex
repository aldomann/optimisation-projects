%-----------------------------------------------------------------
%	STATISTICAL ANALYSIS
%	!TEX root = ./../main.tex
%-----------------------------------------------------------------
\section{Implementation of the Program}\label{sec:implementation}

\subsection{Preprocessing the Data}
Since the data provided includes a header, we decided the get rid of it using AWK, and change the format to store it in a \inline{.csv} file:
\begin{lstlisting}[language=awk]
BEGIN{
	FS = " "
	OFS = ","
}
NR > 1 && NF == 2 {
	print $1, $2
}
\end{lstlisting}

We just run it as usual:
\begin{lstlisting}[language=bash]
awk -f clean_data.awk KnapsackData.dat > data.csv
\end{lstlisting}


\subsection{Algorithm in Python}
The algorithm in Python 3.5 is programmed mainly using the NumPy library. 

First of all, we use Pandas to read and process the file with the data:
\begin{lstlisting}
data = pd.read_csv('data.dat')
values = np.array(data["values"])
weights = np.array(data["weights"])
num_items = len(data)
\end{lstlisting}


Two important parameters in the algorithm are the total weight and the total value of the items in the truck. For this reason, we have defined the \inline{get_weight()} and \inline{ get_value()} functions:
\begin{lstlisting}
def get_weight(individual, weights):
	return int(np.sum(individual * weights))
\end{lstlisting}

\begin{lstlisting}
def get_value(individual, values, weights):
	total_weight = get_weight(individual, weights)
	if (total_weight <= 600):
		total_value = np.sum(individual * values)
	else:
		total_value = - pow(total_weight, 100)
	return int(total_value)
\end{lstlisting}

Where the \inline{individual} is the array $\va{x}$ array of zeroes and ones used as an indicator of the present items in the truck.


We use a simple Monte Carlo method to mutate the elements $x_{i}$ of the individual:
\begin{lstlisting}
def random_bool():
	return np.random.randint(2)
\end{lstlisting}


For the Simulated Annealing we define a function which is just a fully fledged version of the algorithm we described in \cref{sec:sa}:
\begin{lstlisting}
def simulated_annealing(individual, MAX_ITER):
	for iter in range(0, MAX_ITER):
		# Temperature cooling schedule
		temp[iter] = MAX_TEMP * 0.5 * (tanh(-log(iter+1) + 10) + 1 )

		# Exploring the neighbourhood
		individual_new = np.copy(individual)
		for m in range(0, 5):
			index = np.random.randint(num_items)
			individual_new[index] = random_bool()

		# Total value of items inside the truck
		value_old = get_value(individual, values, weights)
		value_new = get_value(individual_new, values, weights)

		# Acceptance probabilities
		if (value_old < value_new):
			prob[iter] = 1
		else:
			prob[iter] = exp(-(value_old - value_new) / temp[iter])

		# Acceptance rule
		if (float(np.random.random(1)) < prob[iter]):
			individual = np.copy(individual_new)

		# Update total value and weight
		sol_values[iter] = get_value(individual, values, weights)
		sol_weights[iter] = get_weight(individual, weights)
	return individual
\end{lstlisting}

To run the algorithm, and store the final solution we just call the function we defined above:
\begin{lstlisting}
solution = simulated_annealing(individual, MAX_ITER)
\end{lstlisting}

\bigskip
To run the algorithm, we just need to run the following:
\begin{lstlisting}
python knapsack_sa.py
\end{lstlisting}