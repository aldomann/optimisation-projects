%-----------------------------------------------------------------
%	INTRODUCTION
%	!TEX root = ./../main.tex
%-----------------------------------------------------------------
\section{Introduction}

The knapsack problem is a classic and widely studied computational problem in combinatorial optimization. Given a set of items, each with a weight and a value, determine the number of each item to include in a collection so that the total weight is less than or equal to a given limit and the total value is as large as possible. It derives its name from the problem faced by someone who is constrained by a fixed-size knapsack and must fill it with the most valuable items. Because the knapsack problem is a very general problem in combinatorial optimization, it has applications in almost every field. For instance, in economics, the knapsack problem
is analogous to a simple consumption model given a budget constraint. In other words, we choose from a
list of objects to buy, each with a certain utility, subject to the budget constraint. One of the most interesting applications is in solving truck loading problem. We have to load a truck with some goods. The values and the weights of these goods are listed here\footnote{\url{http://mat.uab.cat/~alseda/MasterOpt/KnapsackData.dat}}, and the truck can carry a maximum weight of 600 Kg. We want to maximize the total value of the taken goods.

\begin{table}[H]
\centering
% \resizebox{\textwidth}{!}{%
    \begin{tabular}{cc}
        \toprule
        \toprule
        values & weights \\
        \midrule
        18     & 17      \\
        11     & 20      \\
        15     & 15      \\
        18     & 12      \\
        15     & 13      \\
        7      & 20      \\
        17     & 13      \\
        15     & 13      \\
        22     & 16      \\
        16     & 15      \\
        18     & 14      \\
        10     & 15      \\
        13     & 21      \\
        15     & 13      \\
        16     & 20      \\
        15     & 19      \\
        12     & 7       \\
                \bottomrule
    \end{tabular}
    \hspace{5pt}
    \begin{tabular}{cc}
        \toprule
        \toprule
        values & weights \\
        \midrule
        13     & 11      \\
        12     & 16      \\
        11     & 12      \\
        14     & 14      \\
        17     & 17      \\
        12     & 18      \\
        15     & 16      \\
        7      & 17      \\
        16     & 25      \\
        16     & 27      \\
        12     & 15      \\
        16     & 7       \\
        13     & 18      \\
        16     & 13      \\
        12     & 13      \\
        10     & 24      \\
        11     & 18      \\
        \bottomrule
    \end{tabular}
    \hspace{5pt}
    \begin{tabular}{cc}
        \toprule
        \toprule
        values & weights \\
        \midrule
        13     & 17      \\
        15     & 11      \\
        16     & 18      \\
        15     & 13      \\
        11     & 16      \\
        14     & 14      \\
        21     & 16      \\
        11     & 21      \\
        13     & 17      \\
        10     & 14      \\
        14     & 18      \\
        18     & 16      \\
        17     & 15      \\
        14     & 11      \\
        11     & 23      \\
        15     & 19      \\
        --     & --      \\
        \bottomrule
    \end{tabular}
    \caption{Values and Weights of Goods to Optimize}
    \label{tab:problemdata}
\end{table}